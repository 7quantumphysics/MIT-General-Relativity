\section*{Video 5: The Stress Energy Tensor and the Christoffel Symbol}
\hskip 25pt Continue to imagine a cloud of dust in space where each particle of the dust has a rest mass $m$.  In the rest
frame of the dust element, its rest energy density is the density of the dust in that particular volume of space.
\begin{equation}
  \rho_0\ \circeq\ mn_0
\end{equation}
Now, lets boost into a new frame moving with vecloity $\vec{v}$ with respect to the dust.  The energy density in this boosted
frame is changed.  First, the energy of the individual moving particles gets a $\gamma$ factor from their kinetic energy in
this new frame.  Second, the space (volume) gets contracted, which gives the number density a factor of $\gamma$ as well.
\begin{equation}
  \begin{aligned}
    \rho\ &=\ \gamma m\gamma n_0 \\
    &=\ \gamma^2\rho_0
  \end{aligned}
\end{equation}
If $\rho$ was a 4-vector component, it would not transform with higher order powers of $\gamma$, which means it can't be
part of a Lorentz vector (nor scalar).  One other thing to recognize is that the energy density is built with a combination
of two time-like components, namely the energy component of momentum $\underline{\mathbf{p}}$ and the number density
component of the number vector $\underline{\mathbf{N}}$.  This means that $\rho$ can be written as
\begin{equation}
  \rho\ =\ p^0N^0\ \equiv\ T^{00}
\end{equation}
which implies that it is a part of some tensor.  Let's write this tensor as the tensor product of the two 4-vectors.
\begin{equation}
  \begin{aligned}
    \underline{\underline{\mathbf{T}}}\ &=\ \underline{\mathbf{p}}\ \bigotimes\ \underline{\mathbf{N}} \\
    &=\ mn\ \underline{\mathbf{u}}\ \bigotimes\ \underline{\mathbf{u}} \\
    &=\ \rho\ \underline{\mathbf{u}}\ \bigotimes\ \underline{\mathbf{u}}
  \end{aligned}
\end{equation}
A component of this tensor is written with two indices.
\begin{equation}
  \begin{aligned}
    T^{\mu\nu}\ &=\ p^{\mu}N^{\nu}\\
    &=\ \rho u^{\mu}u^{\nu}
  \end{aligned}
\end{equation}
This can physically be interpreted as the flux of the momentum component $p^{\mu}$ in the $x^{\nu}$ direction.  With
this interpretation, it is worth examaning the components of the tensor this way.
\begin{gather}
  T^{00}\ \equiv\ \mathit{flux\ of\ energy\ density\ in\ time\ "direction"} \\
  T^{0i}\ \equiv\ \mathit{flux\ of\ energy\ density\ in\ the}\ x^i\ \mathit{direction}\
  (\mathit{through\ the}\ x^j\wedge x^k\ \mathit{plane}) \\
  T^{i0}\ \equiv\ \mathit{flux\ of\ momentum\ density\ in\ time\ "direction"} \\
  T^{ij}\ \equiv\ \mathit{flux\ of}\ p^i\ \mathit{in\ the}\ x^j\ \mathit{direction}\ 
  (\mathit{units\ of\ pressure\ for}\ i=j\ \mathit{and\ shear\ for}\ i\neq j)
\end{gather}
There is a symmetry to take note of, which is made apparent by writing it in terms of its components.  I'll choose to
write it in terms of the velocity components.
\begin{gather}
  T^{00}\ =\ \gamma^2\rho_0 \\
  T^{0i}\ =\ \gamma^2\rho_0v^i \\
  T^{i0}\ =\ \gamma^2\rho_0v^i \\
  T^{ij}\ =\ \gamma^2\rho_0v^iv^j
\end{gather}
This tensor $\underline{\underline{\mathbf{T}}}$ is called the Stress, Energy, Momentum tensor.  
The symmetry of $\underline{\underline{\mathbf{T}}}$ has been shown for dust.  In general, this tensor is
symmetric for all known physical structures in space-time (this is not proven here, but I suppose a rigorous
proof can be found if one wants one).  To deduce what $\underline{\underline{\mathbf{T}}}$ looks like, for some material
being studied, we can apply the physical meaning of the tensor's individual components and construct it from there.  Remember,
we extracted the physical meaning of the components $T^{\mu\nu}$ from examining how dust behaves in inertial frames of
reference.

\hskip 25pt For a new example, lets look at a perfect fluid.  This is a fluid where there is no energy flow in
some frame w.r.t. the fluid and no lateral stresses (basically no viscosity).  As a neat "example" of a perfect fluid, one
can think of "dry" water.  The physics of such a fluid will be described by its energy density, $\rho$, and its
pressure, $P$.  Lets also assume that the pressure is isotropic to space (the pressure is the same in all three directions).
In this frame of a perfect fluid, the momentum tensor only has four components.
\begin{equation}
  T^{\mu\nu}\ =\ 
  \begin{pmatrix}
    \rho & 0 & 0 & 0 \\
    0 & P & 0 & 0 \\
    0 & 0 & P & 0 \\
    0 & 0 & 0 & P \\
  \end{pmatrix}
\end{equation}
Picking out the rest frame of the fluid to be the frame of reference in question, where there is no energy flow,
$T^{00}$ is expressed in terms of the density and the four-velocity as the "dust" is described in the previous example.
\begin{equation}
  T^{00}\ =\ \rho\ \underline{\mathbf{u}}\ \bigotimes\ \underline{\mathbf{u}}
\end{equation}
The other three components are diagonal spatial terms which can be expressed with the Minkowski metric times the pressure
term.  The only caveat is that writting the pressure as this multiplication requires that we get rid of the extranuous
"time" component by adding a pressure term to $T^{00}$ we just defined.  (We add it because we are working with a metric
signature where the time element carries a minus sign in the metric)
\begin{gather}
  \underline{\underline{\mathbf{T}}}\ =\
  \left(\rho\ +\ P\right)\
  \underline{\mathbf{u}}\ \bigotimes\ \underline{\mathbf{u}}\ +\ P\underline{\underline{\mathbf{\eta}}} \\
  T^{\mu\nu}\ =\ \left(\rho\ +\ P\right)u^{\mu}u^{\nu}\ +\ P\eta^{\mu\nu}
\end{gather}
At some point we want to be able to express this tensor for a perfect fluid in a general space-time and not just a
Minkowski space-time.  For now, this is good enough for an intuition.

\hskip 25pt Let's consider the physics involving $\underline{\underline{\mathbf{T}}}$.  Think of a cube of
length $l$ submerged in a fluid in space.  We align the faces of the cube such that the normal vectors of the faces of
the cube are parallel with the usual basis of Cartesian unit vectors $\hat{e}^{i}$.  The pressure will be the
force of the fluid acting on each face (surface area) of the cube.  The force on a face of the cube is given classically as
\begin{align*}
  \vec{F}_{\mathit{on\ face\ whose\ normal\ vector\ is}\ \hat{n}}\ &=\ \int\ \vec{P}_n da_n \\
  &=\ \int\
  \left (P_x\hat{x}\ +\ P_y\hat{y}\ +\ P_z\hat{z}\right )_n da_n
\end{align*}
which can be written in terms of components $T^{ij}$.  The area, $a$, of each face is $l^2$.  For each of the 6 faces of
the cube, the force is found to be
\begin{gather}
  \vec{F}\ =\ \mathbf{T}da_n \\
  F^i\ =\ \pm T^{in}l^2;\ n=\{1,\ 2,\ 3\}
\end{gather}
where $-$ represents a pressure acting on the outside of the cube, compressing it, and $+$ represents a pressure of the fluid
flowing from the inside of the cube to the outside, if the cube allows for the fluid to flow through it.
The letter $n$ indicates the Cartesian direction parallel with the normal vector $\hat{n}$ of the face of the cube in
question.  In this fluid, the
net force on the cube is zero.  If the fluid can flow through the cube, the net flux is also zero.  This is expected.  An
interesting way to give a physical argument for the spatial symmetry of $T^{ij}$ is to consider torques and moments of
inertia of this cube, still submersed in this fluid.  Let's assume a axis of rotation to be passing through the center of
the cube, aligned in the $\hat{e}^k$ direction (so the cube can rotate in the $ij$ plane).  Define $\vec{r}$ to point from
the origin to the axis of rotation at the center of the cube and $\vec{r}'$ to point from the origin to a point on the
surface of the cube that will be rotating about the axis.  We can define a new variable,
$\vec{r}_c\ =\ \vec{r}\ -\ \vec{r}'$, so that we can define the torque of a point on a face of the cube.
\begin{equation}
  \vec{\tau}\ =\ \vec{r}_c\ \times\ \vec{F}
\end{equation}
When writing the components of the torque, it's worth noting that $\vec{r}_c$ will always be pointing in the general
direction of the unit normal vector of the face in question ($\vec{r}_c\cdot\hat{n}>0$).  This means that $n$ will be
restricted to which face of the cube you choose to calculate the torque with.  To phrase this in a more direct way,
the number $n$ will equal the index chosen for the component of $\vec{r}_c$ in the calculation.  For example, if the
calculation requires the $i^{th}$ component of $\vec{r}_c$, then $n\equiv i$.  With this in mind, the torque of a
point on a face of the cube about and axis parallel to one of the three Cartesian coordinate axes is written component
wise as follows:
\begin{equation}
  \begin{aligned}
    \tau_k\ &=\ \pm\epsilon_{ijk}r_c^iF^j \\
    &=\ \pm\epsilon_{ijk}r_c^iT^{ji}l^2
  \end{aligned}
\end{equation}
where $l^2$ is the length of the cube squared (as to not confuse this exponent for one of the indices).  So, suppose I
wanted the $z$ component of this torque.  I can write my indices in terms of $x$, $y$ and $z$ now for the sake of
simplicity.
\begin{equation}
  \tau_z\ =\ \pm\left (r_c^xT^{yx}\ -\ r_c^yT^{xy}\right )l^2
\end{equation}
To simplify this further, lets also choose $\vec{r}_c$ such that $r_c^x\ =\ r_c^y\ =\ l/\sqrt{2}$.
\begin{equation}
  \tau_z\ =\ \pm\frac{1}{\sqrt{2}}\left (T^{yx}\ -\ T^{xy}\right )l^3
\end{equation}
This shows that if the tensor is symmetric, then the torque completely vanishes.  Even if it wasn't, the torque would still
rapidly vanish as the size of the cube shrinks to zero.  This isn't surprising.  

\hskip 25pt Finally, lets consider the moment of inertia of this cube about the same axis we defined the torque.  The
inertia is
\begin{equation}
  I\ =\ \int\ \vec{r}_c\cdot\vec{r}_c\ dm
\end{equation}
The exact solution to this integral is not necessary.  What's important to get out of this is the form of the inertia.
\begin{equation}
  I\ \propto\ l^2\rho l^3\ =\ \rho l^5
\end{equation}
This is the cool part of the physics that's been being built towards.  Remember that the torque is related to the
moment of inertia in a similar way that force is related to mass.  Also, notice that as $l\rightarrow0$ the
torque dies off as $l^3$, but the inertia dies off more rapidly as $l^5$.  If we check the angular acceleration of the
cube, we find that it is inversly related to the length of the cube.
\begin{equation}
  \ddot{\varphi}\ \propto\ \frac{T^{yx}\ -\ T^{xy}}{l^2}
\end{equation}
Assuming that the spatial tensor components were not symmetric, this would describe some angular acceleration term that
would exist on small scales $l << 0$.  Imagine if the molecules in a cup of water were to spontanously begin to spin and
rotate faster and faster for no obvious reason other than the stress energy not having symmetric terms.  Because we don't
see this behavior in nature, we can use this as good motivation for saying that the stress energy tensor is symmetric.
This example gives us a physical intuition about the symmetry of the spatial components, but again, another
derivation of the stress energy tensor reveals the symmetry of the tensor in a more rigorous way, in both spatial and
temporal components.  

\hskip 25pt Its important to bring up the conservation of energy and momentum, as a great deal of physics understanding
hinges on these two conservation laws.  In terms of the stress-energy-momentum tensor, this single tensor is a conserved
quantity.  When defining the tensor from a field theory perspective, one can prove the conservation law directly.
\begin{equation}
  \partial_{\mu}T^{\mu\nu}\ =\ 0
\end{equation}
This is similar to Maxwell's Equations in covariant form.
\begin{equation}
  \partial_{\mu}F^{\mu\nu}\ =\ -\mu_0J^{\mu}
\end{equation}
The main difference is that there is no vector "source" term for the variation of the stress-energy-momentum terms.
I suppose if one finds this strange acceleration in nature, this would prove that our theories need to be
re-structured to allow for the non-conservation of energy and momentum!

\hskip 25pt  Let's consider yet another common example of matter; lets consider a point mass, $m_0$ moving along a
world line in space-time.  The world line will be some four-vector of position parametrized by the proper time of the
mass moving along that world line.  The world line will be represented as $\underline{\mathbf{x}}(\tau)$.  To represent
the stress-energy-tensor of this point mass it's beneficial to make use of the delta function and write it in terms of
the mass/energy density.  Indicate the location of the point mass with an apostrophe (').  To start, we know the first
element of the tensor is the same as that of the dust.
\begin{equation}
  T^{00}\ =\ \rho u^0u^0;\ \ [\rho]\ =\ \frac{\mathit{mass}}{\mathit{length}^3}
\end{equation}
For the point mass in Euclidean space the mass density is described as
\begin{equation}
  \rho\ =\ m_0\delta^3\left (\vec{x}\ -\ \vec{x}'\right );\ \ [\rho]\ =\ \frac{\mathit{mass}}{\mathit{length}^3}
\end{equation}
but we want an expression for space-time that will adhere to the correct dimensions for $\rho$.  The space-time
delta function will carry a temporal dimension for any observer in any reference frame.  Again, these space-time
coordinates for the point mass are chosen to be expressed as a function of proper time.  Therefore, this mass
density with the space-time delta function must be integrated over all of "proper" time.  This also maintains the
dimensionality of $\rho$ when given the 4D space-time delta function.
\begin{equation}
  \rho\ =\ m_0\int\ \delta^4\left (\underline{\mathbf{x}}\ -\ \underline{\mathbf{x}}'\right )\ d\tau;\ \ 
  [\rho]\ =\ \frac{\mathit{mass}}{\mathit{length}^3\times\mathit{time}}\times\mathit{time}
\end{equation}
We can express $\underline{\underline{\mathbf{T}}}$ of the point mass the same way we expressed the tensor for dust.  The
only difference is that the Dirac Delta function is serving a role here in place of the number density, since this point
mass is obviously the only mass in question.
\begin{equation}
  T^{\mu\nu}\ =\ m_0\int\ u^{\mu}u^{\nu}\delta^4\left (\underline{\mathbf{x}}\ -\ \underline{\mathbf{x}}'\right )\ d\tau
\end{equation}
As a calculus reminder of integrals with delta functions,
\begin{equation}
  \int\ f(x)\delta\left (g(x)\right )\ dx\ =\ \sum_{i=1}^N\frac{f(x_i)}{g'(x)|_{x=x_i}}\ |_{x_i=x_0}
\end{equation}
where $x_0$ are the $N$ roots of $g(x)$ and $g'(x)$ is the first derivative with respect to $x$.
With this in mind, lets solve the integral over the proper time, choosing one of the four delta functions to work with.
\begin{equation}
  \int\ \left [u^{\mu}u^{\nu}\delta^3
    \left (\vec{x}(\tau)\ -\ \vec{x}(\tau)'\right )\right ]\delta\left (t\ -\ \tau\right )\ d\tau\ =\
  \frac{u^{\mu}u^{\nu}\delta^3\left (\vec{x}(t)\ -\ \vec{x}(t)'\right )}{1}
\end{equation}
Here, this $1$ in the denominator (again, in natural units) is the zeroth component of the mass' 4-velocity in it's frame
of reference.  The final form of the stress-energy-momentum tensor for the point mass can now be written without that
integral.
\begin{equation}
  T^{\mu\nu}\ =\ \frac{m_0}{u^0}u^{\mu}u^{\nu}\delta^3\left (\vec{x}(t)\ -\ \vec{x}(t)'\right )
\end{equation}
To breifly summarize the introduction of the stress-energy-momentum tensor, we derived three examples listed
below.

-----------------------------------------------------------------------------------------------------------------------------
\begin{gather}
  T_{\mathit{non-interacting\ dust}}^{\mu\nu}\ =\ \rho u^{\mu}u^{\nu} \\
  T_{\mathit{perfect\ fluid}}^{\mu\nu}\ =\ \left(\rho\ +\ P\right)u^{\mu}u^{\nu}\ +\ P\ \eta^{\mu\nu} \\
  T_{\mathit{point\ mass}}^{\mu\nu}\ =\ \frac{m_0}{u^0}u^{\mu}u^{\nu}\delta^3\left (\vec{x}(t)\ -\ \vec{x}(t)'\right )
\end{gather}

-----------------------------------------------------------------------------------------------------------------------------

\hskip 25pt Let's consider the classical field equation for Newtonian gravity.  
\begin{equation}
  \nabla^2\Phi\ =\ 4\pi G\rho
\end{equation}
Recall that $\rho$ is just one component of a tensor.  For a better theory of gravity, we should not choose to
pick out this one term out of the other 15 terms of the tensor.  This means we should seek an equation that looks like
this classical differential equation, but with $T^{\mu\nu}$ on the RHS.  This idea will be brought up again later.  We
bring it up because we have at least began to get a feel for the stress-energy-momentum tensor, so we should keep gravity
in mind in how it will be connected with this tensor.

\hskip 25pt As an introduction into curved space-time, let's first start with "simple" Minkowski space-time defined in
curvilinear coordinates.  In traditional Cartesian coordinates, our 4-vector coordinates are given as
$\underline{\mathbf{x}}\ =\ (t,\ x,\ y,\ z)$.  In this basis, the differential of the 4-vector can be written simply as
contractions with the basis vectors with straightforward understanding.
\begin{equation}
  \underline{\mathbf{dx}}\ =\ dx^{\mu}\hat{e}_{\mu}
\end{equation}
The easiest way to switch to curvilinear coordinates is to change to cylinderical coordinates with the usual mapping.  
$\underline{\mathbf{x}}\ =\ (t,\ r,\ \varphi,\ z)$.  In this basis, the differential of the 4-vector is written the same
way, but unit analysis of the angular piece, $dx^2\hat{e}_2\ \equiv\ d\varphi\ \widehat{\varphi}$, shows that the unit vector
$\widehat{\varphi}$ must carry the dimensionality of length to make the displacement vector make sense.  This shows that these
coordinates are not a "straightforward" basis.  In a more math oriented way of putting it, the basis vectors are not
always necessarily normalized.  For the cylindrical coordinates, $\widehat{\varphi}\cdot\widehat{\varphi}\ \neq\ 1$.  With
two different coordinate systems established, its worth understanding how to transform from one set of coordinates to
another in a general sense.  Lets define the transformation from one coordinate system, $\underline{\mathbf{x}}$
(say Cartesian coordinates), to another coordinate system, $\underline{\widetilde{\mathbf{x}}}$ (say cylinderical coordinates).
\begin{equation}
  L^{\alpha}_{\mu}\ =\ \frac{\partial x^{\alpha}}{\partial\tilde{x}^{\mu}}
\end{equation}
For these two specific coordinate systems, the form of $\underline{\underline{\mathbf{L}}}$ can be written as a matrix.
\begin{equation}
  \underline{\underline{\mathbf{L}}}\ =\
  \begin{pmatrix}
    1 & 0 & 0 & 0 \\
    0 & \cos\varphi & \sin\varphi & 0 \\
    0 & -r\sin\varphi & \cos\varphi & 0 \\
    0 & 0 & 0 & 1 
  \end{pmatrix}
\end{equation}
\begin{equation}
  \underline{\underline{\mathbf{L}}}^{-1}\ =\
  \begin{pmatrix}
    1 & 0 & 0 & 0 \\
    0 & \cos\varphi & -\sin\varphi/r & 0 \\
    0 & \sin\varphi & \cos\varphi/r & 0 \\
    0 & 0 & 0 & 1 
  \end{pmatrix}
\end{equation}
