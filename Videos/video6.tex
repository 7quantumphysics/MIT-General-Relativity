\section*{Video 6: The Principle of Equivalence}
\hskip 25pt Continuing where the last video left off, we define the basis vectors for the cylindrical coordinates in terms
of the Cartesian basis.
\begin{equation}
  \begin{aligned}
    \widehat{r}\ &=\ \cos\varphi\ \hat{x}\ +\ \sin\varphi\ \hat{y}\ =\ L^{\alpha}_{1}\ \hat{e}_{\alpha}\\
    \widehat{\varphi}\ &=\ -\sin\varphi\ \hat{x}\ +\ \cos\varphi\ \hat{y}\ =\ L^{\alpha}_{2}\ \hat{e}_{\alpha}
  \end{aligned}
\end{equation}
Now, remember that one of the first tensors to work with is the metric for Minkowski space.
\begin{equation}
  \eta_{\mu\nu}\ =\ \hat{e}_{\mu}\ \cdot\ \hat{e}_{\nu}
\end{equation}
To simply define a general metric, we do the same thing with any coordinate basis.  The general metric is labeled as
$\underline{\underline{\mathbf{g}}}$.
\begin{equation}
  g_{\mu\nu}\ =\ \hat{e}_{\mu}\ \cdot\ \hat{e}_{\nu}
\end{equation}
At some point, we want to calculate derivatives.  With curvilinear coordinate systems in general, the basis vectors
vary with the coordinates.
\begin{gather}
  \frac{\partial\hat{r}}{\partial r}\ =\ 0 \\
  \frac{\partial\hat{r}}{\partial\varphi}\ =\ \frac{\widehat{\varphi}}{r} \\
  \frac{\partial\widehat{\varphi}}{\partial r}\ =\ \frac{\widehat{\varphi}}{r} \\
  \frac{\partial\widehat{\varphi}}{\partial\varphi}\ =\ -r\ \hat{r}
\end{gather}

\hskip 25pt Lets say there is a vector $\underline{\mathbf{V}}$ that is defined in cylindrical coordinates as
$V^{\alpha}\hat{e}_{\alpha}$.  Supoose we wanted to write the "gradient" of this vector.  In an abstract,
ugly notation, this would be
\begin{equation}
  \underline{\nabla}\underline{\mathbf{V}}\ =\
  \partial_{\beta}\left (V^{\alpha}\hat{e}_{\alpha}\right )\widetilde{\omega}^{\beta}
\end{equation}
where $\widetilde{\omega}^{\beta}$ are the basis one-forms ("inverses" of the basis vectors $\hat{e}_{\beta}$).
Ignoring the $\widetilde{\omega}^{\beta}$ for a moment, the rest of this "gradient" should be some "tensorial"
object.  Lets expand that derivative term.
\begin{equation}
  \frac{\partial\underline{\mathbf{V}}}{\partial x^{\beta}}\ =\
  \frac{\partial V^{\alpha}}{\partial x^{\beta}}\hat{e}_{\alpha}\ +\
  V^{\alpha}\frac{\partial\hat{e}_{\alpha}}{\partial x^{\beta}}
\end{equation}
Note that the sum of these two components behaves as a tensor, but the individual addens do not.  With that in the back of our
head, lets look at the second term.  We have shown (at least for cylindrical coordinates) that the derivatives of
basis vectors can be written in terms of other basis vectors.  So, we can choose to write the second term as a
linear combination of basis vectors.  To write it in the Einstein index summation notation, we introduce a new symbol
to help define the coefficients for the linear combination.
\begin{equation}
  \partial_{\beta}\hat{e}_{\alpha}\ =\ \Gamma_{\alpha\beta}^{\mu}\hat{e}_{\mu}
\end{equation}
The Christoffel symbol $\Gamma$ is not a tensor, but it can be used to get tensorial objects.  The
Christoffel symbol is understood with three indices.  The top index shows which unit vector we are using as a basis for
the linear combination, the first lower index shows which basis vector is being differentiated and the
second lower index shows which coordinate we are differentiating with.  For the cylindrical
coordinates, the derivatives of the basis vectors we worked out earlier shows us the "relavent" Christoffel symbols.
\begin{gather}
  \Gamma^{r}_{rr}\ =\ \Gamma^{\varphi}_{rr}\ =\ 0 \\
  \Gamma^{\varphi}_{r\varphi}\ =\ \frac{1}{r} \\
  \Gamma^{\varphi}_{\varphi r}\ =\ \frac{1}{r} \\
  \Gamma^{r}_{\varphi\varphi}\ =\ -r
\end{gather}
Now, the derivative of the 4-vector can be written in terms of the linear combination of basis vectors, or in terms of the
Christoffel symbols.
\begin{equation}
  \partial_{\beta}\underline{\mathbf{V}}\ =\
  \partial_{\beta}V^{\alpha}\hat{e}_{\alpha}\ +\ V^{\alpha}\Gamma^{\mu}_{\alpha\beta}\hat{e}_{\mu}
\end{equation}
Taking notice of the dummy indices required for the contractions, we can factor out the basis vectors $\hat{e}_{\alpha}$.
\begin{equation}
  \partial_{\beta}\underline{\mathbf{V}}\ =\
  \left (\partial_{\beta}V^{\alpha}\ +\ V^{\mu}\Gamma^{\alpha}_{\mu\beta}\right )\hat{e}_{\alpha}
\end{equation}
The term in the parenthesis comes up A LOT, so lets give it some new notation and a fancy name to boot!  It's called
the covariant derivative.  So, the term in the parenthesis is now the covariant derivative of component $V^{\alpha}$
with respect to $x^{\beta}$ and is labeled as $\nabla_{\beta}V^{\alpha}$.
\begin{equation}
  \begin{gathered}
    \partial_{\beta}\underline{\mathbf{V}}\ =\
    \left (\nabla_{\beta}V^{\alpha}\right )\hat{e}_{\alpha} \\
    \nabla_{\beta}V^{\alpha}\ =\ \partial_{\beta}V^{\alpha}\ +\ \Gamma^{\alpha}_{\mu\beta}V^{\mu}
  \end{gathered}
\end{equation}
An immediate application for the covarient derivative can be for the divergence of a 4-vector.
\begin{equation}
  \underline{\mathbf{\nabla}}\cdot\underline{\mathbf{V}}\ =\ \nabla_{\alpha}V^{\alpha}\ =\
  \partial_{\alpha}V^{\alpha}\ +\ \Gamma^{\alpha}_{\beta\alpha}V^{\beta}
\end{equation}
For the cylindrical coordinates this has five terms.
\begin{equation}
  \underline{\mathbf{\nabla}}\cdot\underline{\mathbf{V}}\ =\
  \partial_{t}V^{t}\ +\ \partial_{r}V^{r}\ +\ \partial_{\varphi}V^{\varphi}\ +\ \partial_{z}V^{z}\ +\ \frac{1}{r}V^{r}
\end{equation}

\hskip 25pt What about taking derivatives of things other than 4-vectors?  For scalars, they have no unit basis vectors by
definition, so they're straightforward.  How about a 1-form?  For a 1-form, let's start with our understanding of
covariant derivatives for scalars and 4-vectors, combined with the knowledge that the contraction of a 4-vector with a
1-form results in a Lorentz scalar.
\begin{equation}
  \begin{aligned}
    \nabla_{\rho}\left (W_{\alpha}V^{\alpha}\right )\ &=\ \partial_{\rho}\left (W_{\alpha}V^{\alpha}\right ) \\
    \nabla_{\rho}W_{\alpha}V^{\alpha}\ +\ W_{\alpha}\nabla_{\rho}V^{\alpha}\ &=\
    \partial_{\rho}W_{\alpha}V^{\alpha}\ +\ W_{\alpha}\partial_{\rho}V^{\alpha} \\
    \nabla_{\rho}W_{\alpha}V^{\alpha}\ +\ W_{\alpha}\nabla_{\rho}V^{\alpha}\ &=\
    \partial_{\rho}W_{\alpha}V^{\alpha}\ +\ W_{\alpha}\nabla_{\rho}V^{\alpha}\ -\ W_{\alpha}\Gamma^{\alpha}_{\mu\rho}V^{\mu} \\
    \nabla_{\rho}W_{\alpha}V^{\alpha}\ +\ \cancelto{}{W_{\alpha}\nabla_{\rho}V^{\alpha}}\ &=\
    \partial_{\rho}W_{\alpha}V^{\alpha}\ +\ \cancelto{}{W_{\alpha}\nabla_{\rho}V^{\alpha}}\
    -\ W_{\alpha}\Gamma^{\alpha}_{\mu\rho}V^{\mu} \\
    \nabla_{\rho}W_{\alpha}V^{\alpha}\ &=\ \partial_{\rho}W_{\alpha}V^{\alpha}\ -\
    W_{\alpha}\Gamma^{\alpha}_{\mu\rho}V^{\mu} \\
    \left (\nabla_{\rho}W_{\alpha}\right )\cancelto{}{V^{\alpha}}\ &=\
    \left (\partial_{\rho}W_{\alpha}\ -\ W_{\mu}\Gamma^{\mu}_{\alpha\rho}\right )\cancelto{}{V^{\alpha}} \\
    \nabla_{\rho}W_{\alpha}\ &=\ \partial_{\rho}W_{\alpha}\ -\ \Gamma^{\mu}_{\alpha\rho}W_{\mu} 
  \end{aligned}
\end{equation}
For tensors we can deduce how to write it based on index balancing and whether or not the index is
"upstairs" or "downstairs".  This is not "rigorous", but it gives enough intuition to determine how many terms we need
(an additional Christoffel symbol per index) and which sign it needs ($+$ for upstairs and $-$ for downstairs).  
\begin{gather}
  \nabla_{\rho}\Phi\ =\ \partial_{\rho}\Phi \\
  \nabla_{\rho}V^{\alpha}\ =\ \partial_{\rho}V^{\alpha}\ +\ \Gamma^{\alpha}_{\sigma\rho}V^{\sigma} \\
  \nabla_{\rho}W_{\beta}\ =\ \partial_{\rho}W_{\beta}\ -\ \Gamma^{\sigma}_{\beta\rho}W_{\sigma} \\
  \nabla_{\rho}T^{\alpha\beta...\mu}_{\gamma\delta...\nu}\ =\
  \partial_{\rho}T^{\alpha\beta...\mu}_{\gamma\delta...\nu}\ 
  +\ \left (\Gamma^{\alpha}_{\sigma\rho}T^{\sigma\beta...\mu}_{\gamma\delta...\nu}\ +\
  \Gamma^{\beta}_{\sigma\rho}T^{\alpha\sigma...\mu}_{\gamma\delta...\nu}\ +\ ...\ +\ 
  \Gamma^{\mu}_{\sigma\rho}T^{\alpha\beta...\sigma}_{\gamma\delta...\nu}\right )\
  -\ \left (\Gamma^{\sigma}_{\gamma\rho}T^{\alpha\beta...\mu}_{\sigma\delta...\nu}\ +\
  \Gamma^{\sigma}_{\delta\rho}T^{\alpha\beta...\mu}_{\gamma\sigma...\nu}\ +\ ...\ +\
  \Gamma^{\sigma}_{\nu\rho}T^{\alpha\beta...\mu}_{\gamma\delta...\sigma}\right )
\end{gather}

\hskip 25pt This information will help to define a better way of finding Christoffel symbols than building the symbols
piece by piece, solving the derivatives of the basis vectors one by one.  There are still some features of tensors and
derivatives that need to be seen before that "better way" can be shown.  To ease our way into it, lets think of a
strange operator acting on a scalar.  Namely, what would the "double gradient" of a scalar look like in a Cartesian
basis?
\begin{equation}
  \underline{\nabla}\underline{\nabla}\Phi\ =\
  \partial_{\alpha}\partial_{\beta}\Phi\ \widetilde{\omega}^{\alpha}\widetilde{\omega}^{\beta}
\end{equation}
Its clear that these partial derivatives are symmetric and can be swapped.  This means that the "tensorial" object
this creates is symmertrical in indices $\alpha$ and $\beta$.  What about a general representation of this "double
gradient"?  The only difference is that it will be dependent on covarient derivatives.
\begin{equation}
  \underline{\nabla}\underline{\nabla}\Phi\ =\
  \nabla_{\alpha}\nabla_{\beta}\Phi\ \widetilde{\omega}^{\alpha}\widetilde{\omega}^{\beta}
\end{equation}
One thing to realize about this is that if this double gradient is a tensor then, regardless of which coordinate
representation we choose, it should be symmetric, because it is already shown to be obviously symmetric under the
Cartesian coordinate basis.  Given that this tensor should be symmetric means that the order of operations of
two covarient derivatives can be swapped, just like partial derivatives.
\begin{equation}
  \nabla_{\alpha}\nabla_{\beta}\Phi\ =\ \nabla_{\beta}\nabla_{\alpha}\Phi
\end{equation}
Lets expand these covariant derivatives out.
\begin{equation}
  \begin{aligned}
    \nabla_{\alpha}\left (\partial_{\beta}\Phi\right )\ &=\ \nabla_{\beta}\left (\partial_{\alpha}\Phi\right ) \\
    \cancelto{}{\partial_{\alpha}\left (\partial_{\beta}\Phi\right )}\
    -\ \Gamma^{\sigma}_{\beta\alpha}\left (\partial_{\sigma}\Phi\right )\ &=\
    \cancelto{}{\partial_{\beta}\left (\partial_{\alpha}\Phi\right )}\
    -\ \Gamma^{\sigma}_{\alpha\beta}\left (\partial_{\sigma}\Phi\right ) \\
    \left (\Gamma^{\sigma}_{\alpha\beta}\ -\ \Gamma^{\sigma}_{\beta\alpha}\right )\partial_{\sigma}\Phi\ &=\ 0
  \end{aligned}
\end{equation}
This reveals something about the Christoffel symbols: If we require that this tensorial operation be symmetric in all
representations as a true tensor, the Christoffel symbols can also be shown to be symmetric given that the
single gradient of the scalar is non-zero.

\hskip 25pt In light of seeing symmetric vs anti-symmetric tensors it's useful now to introduce some more notation.  We
can define "simplified" notation to express a "symmetric" operation and an "anti-symmetric" operation of a tensor.
\begin{gather}
  T_{\{\alpha\beta\}}\ \equiv\ \frac{1}{2}\left (T_{\alpha\beta}\ +\ T_{\beta\alpha}\right ) \\
  T_{[\alpha\beta]}\ \equiv\ \frac{1}{2}\left (T_{\alpha\beta}\ -\ T_{\beta\alpha}\right )
\end{gather}
Given what we've just shown for the Christoffel symbols, we can write some expressions.
\begin{gather}
  \Gamma^{\sigma}_{\{\alpha\beta\}}\ =\ \Gamma^{\sigma}_{\alpha\beta} \\
  \Gamma^{\sigma}_{[\alpha\beta]}\ =\ 0
\end{gather}
We can also remember what happens when a symmetric matrix gets contracted with an anti-symmetric matrix,
$\underline{\underline{\mathbf{A}}}$.
\begin{equation}
  \Gamma^{\sigma}_{\alpha\beta}A^{\alpha\beta}\ =\
  \Gamma^{\sigma}_{\{\alpha\beta\}}A^{[\alpha\beta]}\ =\ 0
\end{equation}

\hskip 25pt With all of this in mind, lets finally derive the "better way" to get Christoffel symbols.  We begin the
derivation with the gradient of the metric tensor.
\begin{equation}
  \underline{\nabla}\ \underline{\underline{\mathbf{g}}}\ =\
  \nabla_{\gamma}g_{\alpha\beta}\ 
  \widetilde{\omega}^{\beta}\ \bigotimes\ \widetilde{\omega}^{\alpha}\ \bigotimes\ \widetilde{\omega}^{\gamma}
\end{equation}
Again, we assert that we want tensorial operations like this to be the same in any given coordinate representation.  So,
this gradient of a general metric should be the same as the gradient of the Minkowski metric.
\begin{equation}
  \underline{\nabla}\ \underline{\underline{\eta}}\ =\
  \nabla_{\gamma}\eta_{\alpha\beta}\ 
  \widetilde{\omega}^{\beta}\ \bigotimes\ \widetilde{\omega}^{\alpha}\ \bigotimes\ \widetilde{\omega}^{\gamma}\ \equiv\ 0
\end{equation}
This leads us to require that the covariant derivative of any general metric must be equal to zero.
\begin{equation}
  \nabla_{\gamma}g_{\alpha\beta}\ =\ 0
\end{equation}
With this choice of three indices ($\gamma$, $\alpha$ and $\beta$) this metric relation can also be expressed
by strategic swapping of these indices.  For this derivation, we shuffle the indices in the same way we shuffle
indices for the Levi-Civita symbol that keep them equal to each other.  For example:
$$ \varepsilon_{abc}\ =\ \varepsilon_{cab}\ =\ \varepsilon_{bca} $$
Also, for each of these three ways to express this metric relation, we expand the definition of the covariant derivative
with the Christoffel symbols.  This is what will lead into the "better" form of the Christoffel symbol.
\begin{equation}
  \nabla_{\gamma}g_{\alpha\beta}\ =\ \partial_{\gamma}g_{\alpha\beta}\ -\
  \Gamma^{\mu}_{\alpha\gamma}g_{\mu\beta}\ -\ \Gamma^{\mu}_{\beta\gamma}g_{\alpha\mu}\ =\ 0
  \label{eq:ONE}
\end{equation}
\begin{equation}
  \nabla_{\beta}g_{\gamma\alpha}\ =\ \partial_{\beta}g_{\gamma\alpha}\ -\
  \Gamma^{\mu}_{\gamma\beta}g_{\mu\alpha}\ -\ \Gamma^{\mu}_{\alpha\beta}g_{\gamma\mu}\ =\ 0
  \label{eq:TWO}
\end{equation}
\begin{equation}
  \nabla_{\alpha}g_{\beta\gamma}\ =\ \partial_{\alpha}g_{\beta\gamma}\ -\
  \Gamma^{\mu}_{\beta\alpha}g_{\mu\gamma}\ -\ \Gamma^{\mu}_{\gamma\alpha}g_{\beta\mu}\ =\ 0
  \label{eq:THREE}
\end{equation}
All of these covariant derivatives are equal to zero, hence all of these expressions are equal to each other;
Expression~\eqref{eq:ONE} = Expression~\eqref{eq:TWO} = Expression~\eqref{eq:THREE}.  This means that
Expression~\eqref{eq:ONE} - Expression~\eqref{eq:TWO} = Expression~\eqref{eq:THREE}.  This ALSO means that
Expression~\eqref{eq:ONE} - Expression~\eqref{eq:TWO} - Expression~\eqref{eq:THREE} = 0.  Without much thought, these
seem to be ugly, arbitrary computations.  But, by taking advantage of the symmetry of the Christoffel symbols, many terms
cancel from this combination of the expressions and leaves behind one Christoffel symbol and a bunch of metric terms.
Without writing all of the cancellations and simplifications explicity, this gives the "better way" of expressing the
Christoffel symbols.
\begin{equation}
  \Gamma^{\mu}_{\alpha\beta}\ =\ -\frac{1}{2}g^{\mu\gamma}
  \left (\partial_{\gamma}g_{\alpha\beta}\ -\ \partial_{\beta}g_{\gamma\alpha}\ -\ \partial_{\alpha}g_{\beta\gamma}\right )
\end{equation}
