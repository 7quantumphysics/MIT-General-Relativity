\section*{Video 8: Lie Transport, Killing Vectors, Tensor Densities}

\hskip 25pt What has been shown in the last video is that the partial derivative of a tensor does not produce another
tensor, which is what we expect from several different arguments: visually, algebraically and remembering the necessity
of the Cristoffel symbols for appropriate derivatives of tensors to account for general curvilinear coordinate bases.  A
proper way to differentiate the vector field between two close points is to parallel transport one vector to the other and
then perform the differentiation as you'd expect.  Using the same example as the previous video (section), let's transport
the vector from point $P$ to point $Q$.  The parallel transport of the vector is done to preserve its tensorial nature.
Lets assume that we can define an object $\mathfrak{T}^{\alpha}_{\beta\mu}$ that allows us to write the transported vector
as a difference between the vector at the original point and the infinitesimal distance each component of the vector was moved
to reach the next point.  We will define this transported vector from $P$ to $Q$ as $A_T^{\alpha}(P\rightarrow Q)$
\begin{equation}
  A_T^{\alpha}\left(P\rightarrow Q\right)\ =\ A^{\alpha}\left(P\right)\ -\
  \mathfrak{T}^{\alpha}_{\beta\mu}dx^{\beta}A^{\mu}
\end{equation}
With the vector "transported", the derivative, which we will label with a capital letter D, is now straightforward.
\begin{equation}
  \begin{aligned}
    \mathfrak{D}_{\beta}A^{\alpha}\ &\equiv\
    \lim_{dx^{\beta}\rightarrow0}
    \left\{\frac{A^{\alpha}\left(Q\right)\ -\ A_T^{\alpha}\left(P\rightarrow Q\right)}{dx^{\beta}}\right\}\\
    &=\ \partial_{\beta}A^{\alpha}\ +\ \mathfrak{T}^{\alpha}_{\beta\mu}A^{\mu}
  \end{aligned}
\end{equation}
In this notation $\mathfrak{T}^{\alpha}_{\beta\mu}$, in general, is known as "the connection" term.
